%-----------------ページ余白の設定---------------------
\usepackage[top=1.5cm, bottom=1.0cm, left=1.4cm, right=1.4cm, includefoot]{geometry}
%------------------パッケージ読み込み--------------------
\usepackage{caption}
\usepackage{graphicx}
\usepackage{amsmath,amssymb,amsthm}				%数式align環境,数学記号,定義・定理・証明環境
\allowdisplaybreaks % align環境の途中での改ページ許可
\usepackage[subrefformat=parens,labelformat=parens]{subcaption}			%複数図をまとめる
\usepackage{tabularray}
\usepackage{url}
% \usepackage{ascmac}				% 囲み
% \usepackage{color}					%色付け
% \usepackage{autonum} % 数式の番号管理
% \usepackage{relsize} % 拡大縮小
\usepackage{float} % 図の位置をその場に指定
    % \restylefloat{figure}
    % \restylefloat{table}
\usepackage{wrapfig} % 図の回り込み
% \usepackage{natbib} 
% \usepackage[sectionbib]{chapterbib}
% \usepackage{showkeys} % ラベル表示


%-------------------コマンドの定義-------------------------
\newcommand{\bm}{\boldsymbol}
\newcommand{\del}{\partial}
\newcommand{\diag}{\mathrm{diag}}

\theoremstyle{definition}
\newtheorem{definition}{Definition}%[section]
\newtheorem{assumption}{Assumption}%[section]
\newtheorem{theorem}{Theorem}%[section]
\newtheorem{lemma}{Lemma}%[section]
\newtheorem{corollary}{Corollary}%[section]
\newtheorem{remark}{Remark}%[section]
% \renewcommand{\proofname}{\textbf{証明}}

% subcaptionの調整
\captionsetup[subfigure]{labelformat=simple}
\renewcommand{\thesubfigure}{(\alph{subfigure})}


%-------------------ラベル表示の変更-----------------------------
\date{\number\year/\number\month/\number\day}
% \renewcommand{\figurename}{Fig.}
% \renewcommand{\tablename}{Table}
% \renewcommand{\refname}{References}
% \bibliographystyle{sicetran}
% \bibliographystyle{IEEEtran}
%\renewcommand{\theenumi}{\alph{enumi}}

\bibliographystyle{ssice}

% タイトルのスタイル
\makeatletter
\renewcommand{\maketitle}{
  \begin{center}
    {\huge\bfseries\@title} \\
    {\large{\@date}\quad{\@author}}
    % \vspace{10pt}
    \hrule
  \end{center}
}
\makeatother
